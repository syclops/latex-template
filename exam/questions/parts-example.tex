\titledquestion{\thequestion. Parts}[10]

This is an example of a question with multiple parts. To avoid double-counting
points in the grading table, you must start the \texttt{parts} environment with
the \texttt{noaddpoints} command and end with the \texttt{addpoints} command.
You are thus responsible for ensuring that the point totals add up correctly.

\noaddpoints

\begin{parts}

  \part[3] This is a sample part.

  \begin{solution}

    This is a solution to a part. It goes in its own \texttt{solution}
    environment.

  \end{solution}

  \part[3] Now we will take a look at how a subpart works.

  \begin{subparts}

    \subpart[1] This is a sample subpart.

    \begin{solution}

      This is a solution to a subpart. It also goes in its own
      \texttt{solution} environment.

    \end{solution}

    \subpart[2] We can also have subsubparts, but let's not worry about that for
    now.

    \begin{solution}

      For any solution just place it in its own \texttt{solution} environment
      after the question text.

    \end{solution}

  \end{subparts}

\end{parts}

\addpoints
